\documentclass[11pt]{article}
\usepackage{graphicx}
\usepackage{amssymb}
\usepackage{epstopdf}
\usepackage{amsfonts}
\usepackage{natbib}
\usepackage{subfigure}
\usepackage{pdfsync}
\usepackage{xspace}
\usepackage{color}
\usepackage{url}
%\usepackage[pdftex, colorlinks=true,urlcolor=blue]{hyperref}
%\usepackage{wallpaper}

%%%% HERE IS SOME WATERMARK STUFF
%\addtolength{\wpXoffset}{-.2in}
%\CenterWallPaper{1.1}{/Users/eriq/Documents/work/nonprj/WaterMarks/StrictDraftWatermark.eps}


\DeclareGraphicsRule{.tif}{png}{.png}{`convert #1 `dirname #1`/`basename #1 .tif`.png}

%% My baseurl.  Can be changed as necessary.


%% some handy things for making bold math
\def\bm#1{\mathpalette\bmstyle{#1}}
\def\bmstyle#1#2{\mbox{\boldmath$#1#2$}}
\newcommand{\thh}{^\mathrm{th}}


%% Some pretty etc.'s, etc...
\newcommand{\cf}{{\em cf.}\xspace }
\newcommand{\eg}{{\em e.g.},\xspace }
\newcommand{\ie}{{\em i.e.},\xspace }
\newcommand{\etal}{{\em et al.}\ }
\newcommand{\etc}{{\em etc.}\@\xspace}


\newcommand{\CR}{\mathrm{CR}}
\newcommand{\CWT}{\mathrm{CWT}}
\newcommand{\PBT}{\mathrm{PBT}}
\newcommand{\Var}{\mathrm{Var}}
\newcommand{\btheta}{\bm{\theta}}

%% the page dimensions from TeXShop's default---very nice
\textwidth = 6.5 in
\textheight = 9 in
\oddsidemargin = 0.0 in
\evensidemargin = 0.0 in
\topmargin = 0.0 in
\headheight = 0.0 in
\headsep = 0.0 in
\parskip = 0.2in
\parindent = 0.0in


\title{A simulation framework to compare \\
the information gained from \\
PBT relative to CWTs}
\author{Eric C. Anderson\thanks{
    Fisheries Ecology Division, 
    Southwest Fisheries Science Center, 
    110 Shaffer Road,
    Santa Cruz, CA 95060}
}
\begin{document}

\maketitle

\begin{abstract}
\noindent I am taking the lead on this aspect of the project:
\begin{quote}
{\sl Assessment of the degree to which this system could or could not deliver estimates of the key life history and fishery parameters that are currently delivered from the CWT program and do so with similar or better accuracy (i.e., consider errors of estimation). Identify areas or issues where implementation of PBT on a coast-wide basis seems most problematic. }
\end{quote}
I realized that I wouldn't feel comfortable recommending (or not recommending) PBT until I had made
a solid effort to get a handle on how PBT's effectiveness is affected by mass-marking and the move to
electronic detection in fisheries.  The simulations I am developing, described herein, are part of
my effort to do so in a meaningful and transparent fashion.  I would appreciate any feedback from the
committee.
\end{abstract}

% \tableofcontents

\section{Introduction}

To detemine whether PBT can economically 
deliver comparable information (or more information) than what is currently achieved with CWTs,
I propose a simulation of
mixed fisheries upon the entire coast.  By changing the parameters of the
simulation we can investigate release-group-specific recovery rates
under a variety of PBT tagging rates, marking protocols, and sampling scenarios, and compare these
to the rate of CWT recoveries achieved under the current CWT system.
The simulations, of course, are driven by real data, and incorporate the effects of mass-marking and the use
(or not) of electronic detection.

This document has four parts (only the first two are near completion this point).
In the first, I discuss the overall idea and character of the simulations
and detail some of the assumptions made in order to render them both meaningful and manageable.  
Performing this type of simulation
requires parameterization with realistic values of the  relative frequency of fish from each release group
in numerous fisheries as well as the fraction of mass-marked fish from unassociated releases, and 
of unmarked fish from wild populations or unassociated releases. Thus, in the second section I document
a Bayesian method developed to use CWT data from the RMIS data base to estimate proportions of fish from
different release groups in different fishery strata.  The third section shows the results of applying this
estimation method to CWT data in a small number of recent years, and the fourth section describes the results
of the simulations and details the comparative performance of PBT vs. CWTs.


\section{Simulation Rationale and Overview}

The relative utility of PBT versus CWTs depends on a large
number of factors which differ between release groups, fisheries, locations and sampling methods.
There are some scenarios where the ability to inexpensively tag 100\% of every release group via PBT
will clearly be advantageous. But there are other scenarios in which CWTs might have the advantage, for example,
cases where electronic detection of fish carrying CWTs
might allow CWT recoveries to be selectively enriched for weak stocks of interest.  In order to assess the relative merits of PBT versus CWTs at a coastwide
scale, it is my belief that a careful accounting of the information obtained from PBT versus
CWT for each release group across the entire coast is necessary.  This seems, at first, to be  monumental
task, but the data for doing this exist in the data bases of the RMIS---there is a rich history of
CWT releases and subsequent catch-sampling and CWT recoveries that can be used to parameterize simulations
in which one can vary PBT and CWT tagging rates and monitor the observed number of recoveries.



The utility of PBT data versus CWT data for any particular release group in any 
particular fishery stratum subject to sampling by any particular protocol (electronic detection, or 
visual, etc.) can be measured by the number of fish from that release group sampled and
recovered from the fishery.  The number of tag recoveries for each release group is thus what
will be our ``currency'' with which we compare the efficacy (and when divided by cost, the cost-effectiveness)
of PBT versus CWTs\footnote{If the uncertainty in the estimated PBT tagging rate was much higher than the
uncertainty in tag rate for any particular release group, then the number of PBT recoveries versus the number of CWT recoveries
would not provide a direct comparison.  However, in previous work, which I can put in an appendix, I've shown
that variance in PBT tagging rates are not a large factor---especially when a large fraction of the fish are
tagged by PBT}.  Certainly, if it were possible to design a PBT system that provided more recoveries
in all fisheries of every release group at the same or less cost than the CWT system, it would be
hard to argue against it.  Of course, the comparison between PBT and CWTs is likely to be
more nuanced and variable across release groups and recovery locations.  For example, PBT may be better
for monitoring some release groups in some fisheries and CWTs better in others. The question I wish to address is,
``Specifically, {\em where} will PBT be better and where will CWTs be better?" and, of course, in the end,
``Averaged over all the locations and release groups along the coast, does any one method appear
better and more cost-effective?''  


These questions could be answered quite directly for every catch sample if the composition of the
fishery from which that catch sample was to be taken was known.  If we knew the percentage of each
release group in the fishery, then we could compute the fraction of the fishery that were ad-clipped
(using the mark rates for each release group) and the fraction that were carrying CWTs (from the
tag rates for each release group), which would allow us to easily simulate CWT recoveries under
visual or electronic sampling.  By varying the tag rates to reflect what would be achieved with
PBT, and by varying how sampling would proceed (for instance by assuming that electronic sampling was not 
available because PBT-tagged fish were not carrying any wire), it would be
relatively straightforward to simulate PBT recoveries in the same fishery, giving us an objective basis upon which
to compare the two systems.  

Unfortunately, the composition of every fishery is not known.  Instead of knowing the truth
and using it, we will have to use existing CWT data to obtain reasonable estimates of
fishery composition, and then use those to drive our simulations. Section~\ref{sec:estimation} describes a Bayesian method for
estimating those compositions from CWT data, providing samples of the mixed fishery proportions
from their posterior distribution which can then be used in the simulations.  


\subsection{Some assumptions / qualifying statements}

In order to make this undertaking manageable, while still yielding useful information, I make
a few assumptions, listed below:
\begin{enumerate}
\item {\sl The tag-code is the basic unit for fishery estimation.} I suspect that for many estimation purposes
it is customary to aggregate multiple tag-codes (release groups) from a given hatchery or across multiple
hatcheries in a single stock, but I have not done that here.  Our guidance has included the fact that a 
PBT system must accommodate multiple release groups within any hatchery, so
there seems a good precedent for focusing on tag-codes as the basic unit for estimation.
\item {\sl Fishery samples across nearby locations and over time can be aggregated. } Here I am basically assuming
that the composition of fisheries across nearby locations and over a year can be regarded as identical.  This is
necessary because very few fishery strata contain enough coded wire tag recoveries to yield a reliable estimate
of the proportion of fish from each release group in the fishery. {\bf It would be very helpful if the technical
advisory committee were able to provide a list of how
different agencies aggregate fishery location and time strata for their management purposes}.
Not knowing that myself, I have just agglomerated spatially
proximate fishery locations over entire years to achieve samples of 500 or more CWTs.  Obviously, this is 
flexible and could be changed with input from the committee.   While this will not
reflect compositional changes over seasons, my feeling is
that we are after a reasonable approximation of what the true composition is, and this will likely
suffice.
\item {\sl The tagged and marked fractions of a release group are constant over the life of the fish.}  This
assumption would be violated if marked or tagged fish experience higher mortality than their unmarked
counterparts, as could be the case in the presence of very strong mark-selective fishing pressure.  Dealing with
that seems like it would be overkill for these simulations.  Remember, we are interested in getting results
for fisheries with proportions that might reasonably be encountered, and I hope that the occurrence of
some extra mortality on tagged fish will not render the proportions used in the simulation ``unreasonable.''
\item {\sl The only type of mark we concern ourselves with in the estimation of proportions is the
adipose clip.}  An adipose clip in conjunction with another mark is just counted simply as an ad-clip.
\item {\sl Currently I am not addressing suitability of either PBT or CWTs for doing DIT-based estimation.}
\item {\sl This is being developed at the moment for recoveries from fisheries, but it could easily be extended to
escapement, \etc}
\end{enumerate}


\section{Bayesian Estimation of Release Group Proportions \label{sec:estimation}}

To parameterize our simulations, we require what we will call $\theta_g$,
the probability that a fish sampled 
from the ocean within a given fishery stratum (the dependence on a given stratum is implied to avoid an
overproliferation of subscripts) is from release group $g$. Note that this might not reflect the
true composition of fish in the ocean because of length restrictions on the catch, \etc  This should
be all right---what we are after is a probability that a fish from a certain release group will
be caught and retained in the fishery, and that is what $\theta_g$ reflects.
For our purposes here, a release group is defined as
a tag code. $\theta_g$ refers not just to the fraction of fish from release group $g$ that
have CWTs, but {\em all} of the fish in the release group $g$, whether they have
a CWT or not. We
will think of tag codes, $g$, as being indexed from 1 up to $G$ for the $G$ possible tag codes
that could turn up in a fishery in the given year. 
We include several other categories:
\begin{description}
\item{$A+$ and $A-$:}~~Fish from non-associated release groups that carry agency only or blank wire.  These fish cannot be identified to release group but they do all carry wire. The $+$ and the $-$ refer to whether or
not the fish is ad-clipped.
\item{$U+$ and $U-$:}~~Fish that don't carry wire of any sort and are not part of an associated release group.  If
they have an ad-clip, they are part of $U+$; if they don't, they are part of $U-$.  Untagged and unmarked wild
fish are part of $U-$, as are unmarked fish from non-associated releases.  Untagged but marked fish from
non-associated releases are in $U+$.
\end{description}
Some portions of the $A$ and $U$ categories are not likely to be easily estimated, as they represent the ``leftover'' after all
the fish from associated releases are accounted for.  It will be interesting to see how their inclusion affects
estimates of composition uncertainty. The parameter we are interested in is thus $\btheta = (\theta_1,\ldots,\theta_{G},
\theta_{A+}, \theta_{A-}, \theta_{U+}, \theta_{U-})$, and we note that 
\[
\theta_{A+} + \theta_{A-} + \theta_{U+} + \theta_{U-} + \sum_{g=1}^{G} \theta_g = 1.
\] 


Each tagged release group $g$ is assumed to have a marked fraction, $f^{(m)}_g$ and an unmarked fraction, $f^{(u)}_g$, 
such that $0 \leq f^{(m)}_g, f^{(u)}_g \leq 1$ and $f^{(m)}_g + f^{(u)}_g = 1$.  We consider a fish to be ``marked'' if it has an ad-clip (whether or
not in combination with another mark).   Associated with each of these fractions is a CWT tagging rate, $p$.  So, $p^{(m)}_g$ denotes the
fraction of the marked fish from release group $g$ that carry CWTs, and $p^{(u)}_g$ denotes the fraction
of unmarked fish from release group $g$ that carry CWTs; $0 \leq p^{(m)}_g, p^{(u)}_g \leq 1$;
however $p^{(m)}_g + p^{(u)}_g$ is not necessarily equal to 1.  For each $g=1,\ldots,G$, the values of the $p$'s and $f$'s can be obtained
from information stored in the RMIS release data base.


It is important to understand that I will be agglomerating different fishery strata in order to obtain
relatively large samples of CWTs with which to estimate $\btheta$.  In such cases I will assume that
$\btheta$ is identical across all agglomerated strata.  This is obviously a simplification, but it does not make 
sense to estimate a separate $\btheta$ for every stratum, some of which have only a few CWT recoveries.
The other consequence
of this policy is that I will likely end up using both visually and electronically detected samples, together, to 
estimate any particular $\btheta$.  Thus, we would like our estimation method to
appropriately weight information from each type of sample.  I believe the Bayesian framework I
describe below achieves that.

\subsection{Missing data formulation}
We can develop our estimation framework by noticing that we are missing information here
that would make the estimation problem quite easy---namely, if we knew the release group of every
fish we sampled in a fishery (whether it had a CWT or an ad-clip or not).  In that case, we would have a simple multinomial
formulation with a Dirichlet prior.  Of course, we don't know the release group of every sampled
fish---that is missing!  But thinking about the problem this way shows us that it is a classic
missing-data problem for which the EM-algorithm or Gibbs sampling are useful.  In our case we will
formulate a way to sample over the unknown origins of all of the sampled fish in a stratum (whether or 
not they have an ad-clip or carry a CWT) given 
everything else we know about them.
In other words, for every fish that is handled in some way (either visually inspected for an ad-clip or
electronically scanned for a CWT) we can sample over that crucial piece of missing data---the fish's origin.
Having sampled all the fish's tag-codes of origin we can use them to update our estimate of
$\btheta$---a simple Gibbs sampler that will be described in detail below.

\subsection{The data (missing and otherwise) for each ``sampled'' fish}

For each fish that is sampled---any fish that someone looks at to see if it has a mark, or
runs  through a metal detector to see if it beeps---the list below shows what we shall keep track of (and
update/impute as necessary).  The names in {\tt monospaced} font below refer to the specifications that
will be used in the computer code that goes with this project, and the mathematical notations immediately
following give
the indicator variables and their values that will be used in the upcoming mathematical formulations.
\begin{description}
	\item [{\tt detection\_method}:] ~~~$\delta_i^{(\mathrm{dm})}$~~~The method used to screen fish to discover CWTs. ($\delta_i^{(\mathrm{dm})}$ is the detection method of fish $i$ in the sample).
	\begin{itemize}
		\item [{\tt V}:] ~~~$\delta_i^{(\mathrm{dm})} = \mathrm{V}$~~~if the fish was visually sampled.
		\item [{\tt E}:] ~~~$\delta_i^{(\mathrm{dm})} = \mathrm{E}$~~~if the fish was electronically sampled.
	\end{itemize}
	\item [{\tt beep}:] ~~~$\delta_i^{(\mathrm{b})}$~~~Whether or not it beeped upon electronic detection. Possible values are:
	\begin{description}
		\item [{\tt yes}:] ~~~$\delta_i^{(\mathrm{b})} = 1$~~~The fish beeped when detected electronically.
		\item [{\tt no}:]  ~~~$\delta_i^{(\mathrm{b})} = 0$~~~The fish did not beep when detected electronically.
		\item [{\tt irrelevant}:] The fish was not sampled electronically.
		Must be irrelevant if detection\_method is {\tt V}. Might be irrelevant if the fish is part of a
		mark sample that is drawn {\em independently} of the beep-status for fish in a fishery with
		electronic detection.
	\end{description}
	\item [{\tt ad\_clipped}:]  ~~~$\delta_i^{(\mathrm{ac})}$~~~Does the fish have an adipose clip? 	An ad-clip, even in combination with any other marks is still counted.
	\begin{description}
		\item [{\tt yes}:]  ~~~$\delta_i^{(\mathrm{ac})} = 1$~~~The fish was observed to have an ad-clip.
		\item [{\tt no}:]  ~~~$\delta_i^{(\mathrm{ac})} = 0$~~~The fish was observed to {\em not} have an ad-clip.
		\item [\textcolor{blue}{{\tt unknown}}:]  ~~~$\delta_i^{(\mathrm{ac})} = \mbox{?}$~~~The fish was either not
		inspected for a mark, or it was and its mark status was ambiguous.
	\end{description}
	\item [{\tt cwt\_status}:] ~~~$\delta_i^{(\mathrm{cs})}$~~~Does the individual carry a CWT or not, and what category of wire is
	it? (this is {\em not} the same as the tag code of the fish).
	\begin{description}
		\item [{\tt cwt}:] ~~~$\delta_i^{(\mathrm{cs})} = 1$~~~The fish is known to carry a coded wire tag, and it
		was read appropriately.
		\item [{\tt no\_read}] ~~~$\delta_i^{(\mathrm{cs})} = n$~~~This corresponds to all fish that
		are observed to have a non-agency-only CWT but which is not readable, or was apparently read incorrectly.  This 
		includes the categories ``Tag lost before read,'' ``Tag not readable,'' 
		and ``Unresolved discrepancy.''
		\item [{\tt awt}:] ~~~$\delta_i^{(\mathrm{cs})} = a$~~~Agency-only or blank wire.
		\item [{\tt no\_tag}:] ~~~$\delta_i^{(\mathrm{cs})} = 0$~~~The fish is observed to {\em not} carry a CWT
		\item [\textcolor{blue}{{\tt unknown}}:] ~~~$\delta_i^{(\mathrm{cs})} = \mbox{?}$~~~It is unknown if the fish carried a CWT or not.  This would be the case, for example
		in fish that had an ad-clip but were not assessed for the presence of a CWT, or fish that did not have
		an ad-clip and were not assessed for the presence of a CWT.
	\end{description}
	\item [{\tt tag\_code}:] ~~~$\delta_i^{(\mathrm{tc})}$~~~What is the tag code that corresponds to the release group that the fish comes from.
	\begin{description}
		\item [{\tt code}:] ~~~$\delta_i^{(\mathrm{tc})} = g$~~~Whatever the tag code is for the release group the fish was a part of.  In fish without CWTs, this value will not be directly known, but we can still think of each
		fish as having an origin.
		use $g$ to index these tag codes.
		\item [\textcolor{blue}{{\tt unknown}}:] ~~~$\delta_i^{(\mathrm{tc})} = \mbox{?}$~~~This is the starting value
		for anything that is a {\tt no\_read}.
		\item [\textcolor{blue}{{\tt pending}}:] ~~~$\delta_i^{(\mathrm{tc})} = p$ ~~~~ This is the value we give things
		that have unknown cwt status.  
		\item [{\tt irrelevant}:] This is the case if {\tt cwt\_status} is not {\tt cwt}, {\tt no\_read}, or {\tt unknown}.  
	\end{description}
\end{description}


The states that constitute ``missing data'' or ``latent variables,'' and which may be updated
are colored blue.

\subsection{MCMC}

Given this setup, we can devise a straightforward Gibbs sampler that samples values of $\btheta$, and the 
{\tt unknown} variables for each sampled fish from their joint posterior distribution. To do so we will
introduce one more parameter whose value will be estimated from the data:  $\alpha$, the probability
of a false-positive beep.  $\alpha$ is the probability that a fish carrying no wire still gets a positive
beep during electronic detection.  We will assume that the false-negative rate for electronic detection is
negligible.

Sampling updated values for $\btheta$, given current values for the missing data is straightforward---the 
full conditional distribution for $\btheta$ is a Dirichlet distribution
with parameters equal to the prior weight on each
release group plus the number of fish from each release group.  Were we using this method to make actual estimates
for informing fishery management, it would probably make sense to spend a lot of time formulating a prior
for the tag codes that used information about proximity or similarity of different release groups. In our
case, however, since we merely want ``reasonable'' estimates for parameterizing a set of simulations, \textcolor{red}{I will assume
the prior on $\btheta$ is a Dirichlet distribution with overall weight determined to give roughly the correct
number of different tag codes (this can be derived from properties of the multivariate Ewens distribution.}

Given a current value for $\btheta$, the values of the {\tt unknown} variables, on all the sampled fish that have
{\tt unknown} values, can be simulated from their full conditional distribution.  For example,
consider a fish, $i$, that was visually-sampled, has an ad-clip, and had a CWT, but it was unreadable.  The 
probability that it is from a release group $g$ can be computed:
\[
P(\delta_i^{(\mathrm{tc})} = g | \delta_i^{(\mathrm{dm})}=\mathrm{V}, \delta_i^{(\mathrm{ac})} = 1, \delta_i^{(\mathrm{cs})} = 1)
\propto
\theta_g f_g^{(m)} p_g^{(m)}
\]
for all $g \in \{1,\ldots,G\}$. Computing this for every $g$ and normalizing it gives the full conditional distribution.

Imputing values for the missing data on fish that are not known to carry a CWT is a two step-process that follows
from decomposing the joint probability of their missing data into the product of two conditional probabilities.
To express this, it will be helpful to describe some marginal probabilities that can be computed from $\btheta$.  The marginal probability that a fish in the stratum has an ad-clip is:
\[
\phi^{(\mathrm{ac})} = \theta_{A+} + \theta_{U+} + \sum_{g=1}^G \theta_g f^{(m)}_g.
\]
The probability that a fish carries a coded wire tag associated with a release group, given it has an ad-clip is:
\[
\phi^{(\mathrm{cs}=1)}_1 =  \sum_{g=1}^G \theta_g f^{(m)}_g p^{(m)}_g,
\]
and given it does not have an ad-clip is:
\[
\phi^{(\mathrm{cs}=1)}_0 =  \sum_{g=1}^G \theta_g f^{(u)}_g p^{(u)}_g,
\]
\etc




\section{Estimation of $\btheta$ in fisheries along the coast}
This is not complete.  I am using R to conduct all of the calculations.  As a geneticist that is coming relatively
recently to the world of CWTs, I've been taking some time to familiarize myself with the RMIS.  All this work
is being done in the open, and can be seen at: \url{https://github.com/eriqande/pbt-feasibility}


\section{Simulation and comparison of PBT and CWTs}

Not yet done.

\bibliography{../bibstuff/pbtfeas}
\bibliographystyle{../bibstuff/men}
 \end{document}
