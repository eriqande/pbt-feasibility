\documentclass[11pt]{article}
\usepackage{graphicx}
\usepackage{amssymb}
\usepackage{epstopdf}
\usepackage{amsfonts}
\usepackage{natbib}
\usepackage{subfigure}
\usepackage{pdfsync}
\usepackage{xspace}
%\usepackage[pdftex, colorlinks=true,urlcolor=blue]{hyperref}
%\usepackage{wallpaper}

%%%% HERE IS SOME WATERMARK STUFF
%\addtolength{\wpXoffset}{-.2in}
%\CenterWallPaper{1.1}{/Users/eriq/Documents/work/nonprj/WaterMarks/StrictDraftWatermark.eps}


\DeclareGraphicsRule{.tif}{png}{.png}{`convert #1 `dirname #1`/`basename #1 .tif`.png}

%% My baseurl.  Can be changed as necessary.


%% some handy things for making bold math
\def\bm#1{\mathpalette\bmstyle{#1}}
\def\bmstyle#1#2{\mbox{\boldmath$#1#2$}}
\newcommand{\thh}{^\mathrm{th}}


%% Some pretty etc.'s, etc...
\newcommand{\cf}{{\em cf.}\xspace }
\newcommand{\eg}{{\em e.g.},\xspace }
\newcommand{\ie}{{\em i.e.},\xspace }
\newcommand{\etal}{{\em et al.}\ }
\newcommand{\etc}{{\em etc.}\@\xspace}


\newcommand{\CR}{\mathrm{CR}}
\newcommand{\CWT}{\mathrm{CWT}}
\newcommand{\PBT}{\mathrm{PBT}}
\newcommand{\Var}{\mathrm{Var}}
\newcommand{\btheta}{\bm{\theta}}

%% the page dimensions from TeXShop's default---very nice
\textwidth = 6.5 in
\textheight = 9 in
\oddsidemargin = 0.0 in
\evensidemargin = 0.0 in
\topmargin = 0.0 in
\headheight = 0.0 in
\headsep = 0.0 in
\parskip = 0.2in
\parindent = 0.0in


\title{A simulation regime to compare the information
gained from PBT relative to CWTs}
\author{Eric C. Anderson\thanks{
    Fisheries Ecology Division, 
    Southwest Fisheries Science Center, 
    110 Shaffer Road,
    Santa Cruz, CA 95060}
}
\begin{document}

\maketitle

\begin{abstract}
\end{abstract}

\tableofcontents

\section{Introduction}

To detemine whether PBT can economically 
deliver comparable information (or more information) than what is currently achieved with CWTs,
I propose a simulation of
mixed fisheries upon the entire coast.  By changing the parameters of the
simulation we can investigate release-group-specific recovery rates
under a variety of PBT tagging rates, marking protocols, and sampling scenarios, and compare these
to the rate of CWT recoveries achieved under the current CWT system.
The simulations, of course, are driven by real data, and incorporate the effects of mass-marking and the use
(or not) of electronic detection.

This document has four parts.  In the first, I discuss the overall idea and character of the simulations
and detail some of the assumptions made in order to render them both meaningful and manageable.  
Performing this type of simulation
requires parameterization with realistic values of the  relative frequency of fish from each release group
in numerous fisheries as well as the fraction of mass-marked fish from unassociated releases, and 
of unmarked fish from wild populations or unassociated releases. Thus, in the second section I document
a Bayesian method developed to use CWT data from the RMIS data base to estimate proportions of fish from
different release groups in different fishery strata.  The third section shows the results of applying this
estimation method to CWT data in a small number of recent years, and the fourth section describes the results
of the simulations and details the comparative performance of PBT vs. CWTs.


\section{Simulation Rationale and Overview}

The relative utility of PBT versus CWTs depends on a very large
number of factors which differ between release groups, fisheries, locations and sampling methods.
There are some scenarios where the ability to inexpensively tag 100\% of every release group via PBT
will clearly be advantageous. But there are other scenarios in which CWTs might have the advantage, for example,
cases where electronic detection of fish carrying CWTs
might allow CWT recoveries to be selectively enriched for weak stocks of interest.  In order to assess the relative merits of PBT versus CWTs at a coastwide
scale, it is my belief that a careful accounting of the information obtained from PBT versus
CWT for each release group across the entire coast is necessary.  This seems, at first, to be  monumental
task, but the data for doing this exist in the data bases of the RMIS---there is a rich history of
CWT releases and subsequent catch-sampling and CWT recoveries that can be used to parameterize simulations
in which one can vary PBT and CWT tagging rates and monitor the observed number of recoveries.



The utility of PBT data versus CWT data for any particular release group in any 
particular fishery stratum subject to sampling by any particular protocol (electronic detection, or 
visual, etc.) can be measured quite simply by the number of fish from that release group sampled and
recovered from the fishery.  The number of tag recoveries for each release group is thus what
will be our ``currency'' with which we compare the efficacy (and when divided by cost, the cost-effectiveness)
of PBT versus CWTs.  Certainly, if it were possible to design a PBT system that provided more recoveries
in all fisheries of every release group at the same or less cost than the CWT system, it would be
hard to argue against it.  The differences between PBT and CWTs are likely to be more variable (PBT better
for monitoring some release groups in some fisheries and CWTs better in others). The question of course is,
``{\em Where} will PBT be better and where will CWTs be better?" and, of course, ``Averaged over the whole
coast, does any one method appear better and more cost-effective?''


These answers could be answered quite directly for every catch-sample if the composition of the
fishery from which that catch sample was to be taken was known.  If we knew the percentage of each
release group in the fishery, then we could compute the fraction of the fishery that were ad-clipped
(using the mark rates for each release group) and the fraction that were carrying CWTs (from the
tag rates for each release group), which would allow us to easily simulate CWT recoveries under
visual or electronic sampling.  By varying the tag rates to reflect what would be achieved with
PBT, and by varying how sampling would proceed if, for instance, electronic sampling was not available because
under a particular PBT scenario fish were not carrying CWTs or even blank wire, then it would be
quite easy to simulate PBT recoveries in the same fishery, giving us an objective basis upon which
to compare the two systems.  

The only ingredient that is missing is the composition of the
fisheries---that is not known. Section~\ref{sec:estimation} describes a Bayesian method for
estimating those compositions from CWT data, providing samples of the mixed fishery proportions
from their posterior distribution which can then be used to drive the simulations.  (Keep in mind,
that what we are really after is a good representation of what those mixed fishery proportions
might be---not necessarily an incredibly accurate estimate of the proportions in any particular
stratum.)

\subsection{Some assumptions / qualifying statements}

In order to make this undertaking manageable, while still yielding useful information, I have made
a few assumptions that I will make explicit here:
\begin{enumerate}
\item {\sl The tag-code is the basic unit for fishery estimation.} I suspect that for many estimation purposes
it is customary to aggregate multiple tag-codes (release groups) from a given hatchery or across multiple
hatcheries in a single stock, but I have not done that here.  The CWT community has been quite outspoken that
PBT will not be useful unless multiple release groups can be maintained within any hatchery, so presumably
there is a good precedent for focusing on tag-codes as the basic unit for estimation.
\item {\sl Fishery samples across nearby locations and over time can be aggregated. } Here I am basically assuming
that the composition of fisheries across nearby locations and over a year can be regarded as identical.  This is
necessary because very few fishery strata contain enough coded wire tag recoveries to yield a reliable estimate
of the proportion of fish from each release group in the fishery. It would be nice to get some input as to how
different agencies aggregate fishery strata; however, in the absence of that I have just agglomerated spatially
proximate fishery locations over entire years to achieve samples of 500 or more CWTs.  Obviously, this is 
flexible and could be changed.  
\item {\sl The tagged and marked fractions of a release group are constant over the life of the fish.}  This
assumption might be violated if marked or tagged fish experience higher mortality than their unmarked
counterparts, as could be the case in the presence of very strong mark-selective fishing pressure.  Dealing with
that seems like it would be overkill for these simulations.  Remember, we are interested in getting results
for fisheries with proportions that might reasonably be encountered, and I don't imagine that the occurrence of
some extra mortality on tagged fish will render the proportions used in the simulation ``unreasonable.''
\item {\sl The only type of mark we concern ourselves with in the estimation of proportions is the
adipose clip.}  An adipose clip in conjunction with another mark is just counted simply as an ad-clip.
\item {\sl Currently I am not addressing suitability of either PBT or CWTs for doing DIT-based estimation.}
\end{enumerate}


\section{Bayesian Estimation of Release Group Proportions}

To parameterize our simulations, we require what we will call $\theta_g$,
the probability that a fish sampled randomly
from the ocean within a given fishery stratum (the dependence on a given stratum is implied to avoid an
overproliferation of subscripts) is from release group $g$.  For our purposes here, a release group is defined as
a tag code, because we have been told
it is essential to be able to track every single release group down to the level of tag code. 
Of course, $\theta_g$ refers not just to the fraction of fish from release group $g$ that
have CWTs, but {\em all} of the fish in the release group $g$, whether they have
a CWT or not. We
will think of tag codes, $g$, as being indexed from 1 up to $G$ for the $G$ possible tag codes
that could turn up in a fishery in the given year. 
We also include two important categories:
\begin{description}
\item{$A$:}~~Fish that are in release groups with tag record status N---\ie some of the fish are carrying blank wire, 
or agency-only pseudo tags and hence, though they can be identified as carrying wire, they cannot be identified to a particular release group.
\item{$U$:}~~all other stocks which are not associated with a release group and which are neither associated
with a release group that has ageny-only wire.  This segment includes wild fish, or hatchery fish that 
are not associated with any metal.   
\end{description}
Note that for the purposes of the analyses I am pursuing, recoveries of pseudo-tags
or agency-only wire will be considered to be fish that are not associated with any tag code (and they
can be either marked or unmarked). Thus, we denote the parameter we are interested in as $\btheta = (\theta_1,\ldots,\theta_{G},
\theta_A, \theta_U)$ and we note that 
\[
\theta_A + \theta_U + \sum_{g=1}^{G} \theta_g = 1.
\] 


Each release group $g$ is assumed to have a marked fraction, $f^{(m)}_g$ and an unmarked fraction, $f^{(u)}_g$, 
such that $0 \leq f^{(m)}_g, f^{(u)}_g \leq 1$ and $f^{(m)}_g + f^{(u)}_g = 1$.  We consider a fish to be ``marked'' if it has an ad-clip (whether or
not in combination with another mark).   Associated with each of these fractions is a CWT tagging rate, $p$.  So, $p^{(m)}_g$ denotes the
fraction of the marked fish from release group $g$ that carry CWTs, and $p^{(u)}_g$ denotes the fraction
of unmarked fish from release group $g$ that carry CWTs; $0 \leq p^{(m)}_g, p^{(u)}_g \leq 1$
and $p^{(m)}_g + p^{(u)}_g = 1$.  For each $g=1,\ldots,G$, the values of the $p$'s and $f$'s can be obtained
from information stored in the releases data base.

In any fishery, the unassociated ($U$) fish are assumed to have a marked and unmarked fraction conditional
on carrying wire and not.  This will typically not be known, but is included in the model and we will
experiment with how much influence the prior on the parameter has.  The same goes for the unknown marked
fraction of the $A$ fish in any fishery.

It is important to understand that I will be agglomerating different fishery strata in order to obtain
relatively large samples of CWTs with which to estimate $\btheta$.  In such cases I will assume that
$\btheta$ is identical across all agglomerated strata.  This is obviously a simplification, but it also
simply does not make sense to estimate a separate $\btheta$ for every stratum.  The other consequence
of this policy is that I will likely end up using both visually and electronically detected samples, together, to estimate any particular $\btheta$.  Thus, we would like our estimation method to
appropriately weight information from each type of sample.  I believe the Bayesian framework I
describe below achieves that.

\subsection{Missing data formulation}
We can develop our estimation framework by noticing that we are missing information here
that would make the estimation problem quite easy---namely, if we knew the release group of every
fish we sampled (whether it had a CWT or an ad-clip or not).  In that case, we would have a simple multinomial
formulation with a Dirichlet prior.  Of course, we don't know the release group of every sampled
fish---that is missing!  But thinking about the problem this way shows us that it is classic
missing-data problem for which the EM-algorithm or Gibbs sampling are useful.  In our case we will
formulate a way to sample over the unknown origins of all of the sampled fish in a stratum (whether or 
not they have an ad-clip or carry a CWT) given 
everything else we know about them.
In other words, for every fish that is handled in some way (either visually inspected for an ad-clip or
electronically scanned for a CWT, we can sample over that crucial piece of missing data---the fish's origin.
Having sampled all the fish's tag-codes of origin we can use the to update our estimate of
$\btheta$---a simple Gibbs sampler that will be described in detail below.

\subsection{The data (missing and otherwise) for each ``sampled'' fish}

For each fish that is sampled (basically someone looks at it to see if it has a mark, or they
run it through a metal detector to see if it beeps, here is what we shall keep track of (and
update/impute as necessary).

\begin{description}
	\item [{\tt detection\_method}]  It gets a ``V'' if it was visually sampled or an ``E'' if electronic.  There should
	not be any NA's here.
	\item [{\tt beep}] yes/no/irrelevant for whether or not it beeped upon electronic detection. Only relevant if detection\_method is E.  Must be irrelevant if
	detection\_method is "V".
	\item [{\tt ad\_clipped}]  yes/no/unknown.  An ad-clip, even in combination with any other marks is still counted
	as just an ad-clip.  unknown means that the fish was either not
	inspected for a mark, or it was and its marks status was ambiguous.
	\item [{\tt cwt\_status}] This stores general information about CWT or other information.  The possibilities are:
	\begin{enumerate}
		\item A tag code (CWT)
		\item Agency-only or blank wire (AWT)
		\item Tag present but not readable (this will be any category like ``tag not readable'', or ``tag
		recovered and then lost,'' etc.) (NoRead)
		\item Not present (i.e. no tag found in the fish)  (NoTag)
		\item NA --- This signifies that it is unknown if the fish actually had a CWT or not.
	\end{enumerate}

\end{description}


We can compute all the possibilities.



I think that pretty much does it for the missing data.  Given an estimate of $\btheta$ and the $f$'s and $p$'s we should be able to compute the joint posterior for all of these possibilities, which means
we can totally sample over them.  Cool!

%% expand.grid(cwt_status = c("CWT", "AWT", "NoRead", "NoTag", NA), ad_clipped = c(TRUE, FALSE, NA), beep = c(TRUE, FALSE), how_sampled = c("V", "E")) %>% rev %>% filter(!(beep == TRUE & how_sampled == "V"))







There are several different sampling scenarios that must be accommodated.
\subsection{Visual sampling only}
In this scenario, $N_t$ fish are inspected in the fishery and $N_m$ of those are found to have
ad-clips (\ie are found to be marked).  Then, a number $s_m$ of the $N_m$ marked fish are
processed for CWTs.  Let $n_g$ be the number of CWTs recovered from release group $g$ in this 
manner, so that, of course, $s_m = \sum_{g=1}^G n_g$. Denote the whole collection of $n_g$'s by $\bm{n}$.  In this case, it appears to me that the
likelihood for $\theta$ can be written down as:
\[
p(N_m, \bm{n} | \btheta) = \psi_M ^ {N_m} \psi_U ^ {N_t - N_m} \prod_{g=1}^G (\theta_g f^{(m)}_g
p^{(m)}_g)^{n_g}
\]
though that is not complete because it needs to have a term for the marked fish with no coded wire
tag.  Hmm...this looks good though.  I think we can deal with all of these thing through Gibbs
sampling of a missing data model where the the missing data is the release group of origin of
each fish.  Clearly, with an estimate of $\btheta$ and knowing the $f$'s and $p$'s we can compute
a full conditional for the origin of each fish in the sample, whether they had a CWT or not...

I think that will be the way forward.



\section{Other}
While $\btheta$ may be what we want to estimate, typically it is easier to estimate quantities
that we will denote by $\phi$ which are conditional probabilities associated with $\btheta$.
For example, consider what we will call $\phi_g^{(m)}$---the fraction of fish, from amongst all those
which carry a CWT, and are marked, that come from release group $g$.   


\section{Boneyard}

It is worth noting that with $\btheta$, and the $f$'s and $p$'s in hand, we ought to be able to
write down a number of probabilities relevant to sampling these fisheries.  For example:
\begin{enumerate}
\item The fraction of marked fish in the fishery.  Call that, $\psi_M$:
\[
\psi_M = \theta_M + \sum_{g=1}^G \theta_g f^{(m)}_g,
\]
\item The fraction of unmarked fish in the fishery, $\psi_U$:
\[
\psi_U = 1- \psi_M = \theta_U + \sum_{g=1}^G \theta_g f^{(u)}_g,
\]
\item The fraction of fish in the fishery that carry CWTs (remember that blank wire does not
count as a CWT):
\[
\psi_C = \sum_{g=1}^G \theta_g(f^{(m)}_g p^{(m)}_g + f^{(u)}_g p^{(u)}_g),
\]
\item The fraction of fish in the fishery that don't carry CWTs:
\[
\psi_{\not{C}} = \theta_M + \theta_U + \sum_{g=1}^G \theta_g[f^{(m)}_g (1 - p^{(m)}_g) + f^{(u)}_g (1-p^{(u)}_g)],
\]
\end{enumerate}
and so on. 



Of course, we are not so much interested in computing probabilities like those listed above as we are 
with {\em estimating} $\btheta$.  I am going to pursue estimation of $\btheta$ from within the Bayesian
framework because I want to take samples from the posterior, and also because it seems that it will
provide a nice, coherent way of dealing with the many different data sources and the variety of 
latent variables in this estimation problem.



\bibliography{../bibstuff/pbtfeas}
\bibliographystyle{../bibstuff/men}
 \end{document}
