\documentclass[11pt]{article}
\usepackage{graphicx}
\usepackage{amssymb}
\usepackage{epstopdf}
\usepackage{amsfonts}
\usepackage{natbib}
\usepackage{subfigure}
\usepackage{pdfsync}
\usepackage{xspace}
%\usepackage[pdftex, colorlinks=true,urlcolor=blue]{hyperref}
%\usepackage{wallpaper}

%%%% HERE IS SOME WATERMARK STUFF
%\addtolength{\wpXoffset}{-.2in}
%\CenterWallPaper{1.1}{/Users/eriq/Documents/work/nonprj/WaterMarks/StrictDraftWatermark.eps}


\DeclareGraphicsRule{.tif}{png}{.png}{`convert #1 `dirname #1`/`basename #1 .tif`.png}

%% My baseurl.  Can be changed as necessary.


%% some handy things for making bold math
\def\bm#1{\mathpalette\bmstyle{#1}}
\def\bmstyle#1#2{\mbox{\boldmath$#1#2$}}
\newcommand{\thh}{^\mathrm{th}}


%% Some pretty etc.'s, etc...
\newcommand{\cf}{{\em cf.}\xspace }
\newcommand{\eg}{{\em e.g.},\xspace }
\newcommand{\ie}{{\em i.e.},\xspace }
\newcommand{\etal}{{\em et al.}\ }
\newcommand{\etc}{{\em etc.}\@\xspace}


\newcommand{\CR}{\mathrm{CR}}
\newcommand{\CWT}{\mathrm{CWT}}
\newcommand{\PBT}{\mathrm{PBT}}
\newcommand{\Var}{\mathrm{Var}}
\newcommand{\btheta}{\bm{\theta}}

%% the page dimensions from TeXShop's default---very nice
\textwidth = 6.5 in
\textheight = 9 in
\oddsidemargin = 0.0 in
\evensidemargin = 0.0 in
\topmargin = 0.0 in
\headheight = 0.0 in
\headsep = 0.0 in
\parskip = 0.2in
\parindent = 0.0in


\title{Estimation of mixed fishery proportions \\
from CWT data}
\author{Eric C. Anderson\thanks{
    Fisheries Ecology Division, 
    Southwest Fisheries Science Center, 
    110 Shaffer Road,
    Santa Cruz, CA 95060}
}
\begin{document}

\maketitle

\begin{abstract}
It seems clear to me that if we are to really get a good sense of whether PBT can
deliver what is currently achieved with CWTs, we are going to have to simulate 
mixed fisheries upon the entire coast and then simulate the sampling within under
a variety of tagging rates of different stocks, etc.  In this way one can compare the
number of recoveries of PBT-tagged fish to those that would be recovered through the use
of CWTs, and we will be able to incorporate the effects of mass-marking and the use
(or not) of electronic detection, \etc  In order to conduct this type of simulation
it will be necessary to know the relative frequency of fish from each CWT-ed stock
in numerous fisheries.  This document documents my approach to this estimation.
\end{abstract}

\section{Introduction}
We are interested in what we will call $\theta_g$, the probability that a fish sampled randomly
from the ocean within a given fishery stratum (the dependence on a given stratum is implied to avoid subscript
overload) is from release group $g$.  For our purposes here, a release group is defined as
a tag code, because the intel we get from the CWT community (whether universally applicable or not) is that
it is essential to be able to track every single release group down to the level of tag code.  We
will think of tag codes, $g$, as being indexed from 1 up to $G$ for the $G$ possible tag codes
that could turn up in a fishery in the given year. 
We also include two important categories:
\begin{description}
\item{$M$:}~~Fish that are ad-clipped ($M$ for ``mass-marked'') but which are not associated
with any tag code.
\item{$U$:}~~all stocks which are not associated with a release group and which are 
{\em unmarked}---In other words, wild fish
that are not tagged, fish from hatcheries that are not CWT-d, and not ad-clipped, etc.)
\end{description}
Note that for the purposes of the analyses I am pursuing, recoveries of pseudo-tags
or agency-only wire will be considered to be fish that are not associated with any tag code (and they
can be either marked or unmarked). Thus, we denote the parameter we are interested in as $\btheta = (\theta_1,\ldots,\theta_{G},
\theta_M, \theta_U)$ and we note that 
\[
\theta_M + \theta_U + \sum_{g=1}^{G} \theta_g = 1.
\] 
Note that for the purposes here, we will consider a fish to be marked if it has an ad-clip (whether or
not in combination with another mark).  For my analyses I will ignore other marks.

Each release group $g$ is assumed to have a marked fraction, $f^{(m)}_g$ and an unmarked fraction, $f^{(u)}_g$, 
such that $0 \leq f^{(m)}_g, f^{(u)}_g \leq 1$ and $f^{(m)}_g + f^{(u)}_g = 1$. Associated with each of these fractions is a CWT tagging rate, $p$.  So, $p^{(m)}_g$ denotes the
fraction of the marked fish from release group $g$ that carry CWTs, and $p^{(u)}_g$ denotes the fraction
of unmarked fish from release group $g$ that carry CWTs; $0 \leq p^{(m)}_g, p^{(u)}_g \leq 1$
and $p^{(m)}_g + p^{(u)}_g = 1$.

It is worth noting that with $\btheta$, and the $f$'s and $p$'s in hand, we ought to be able to
write down a number of probabilities relevant to sampling these fisheries.  For example:
\begin{enumerate}
\item The fraction of marked fish in the fishery.  Call that, $\psi_M$:
\[
\psi_M = \theta_M + \sum_{g=1}^G \theta_g f^{(m)}_g,
\]
\item The fraction of unmarked fish in the fishery, $\psi_U$:
\[
\psi_U = 1- \psi_M = \theta_U + \sum_{g=1}^G \theta_g f^{(u)}_g,
\]
\item The fraction of fish in the fishery that carry CWTs (remember that blank wire does not
count as a CWT):
\[
\psi_C = \sum_{g=1}^G \theta_g(f^{(m)}_g p^{(m)}_g + f^{(u)}_g p^{(u)}_g),
\]
\item The fraction of fish in the fishery that don't carry CWTs:
\[
\psi_{\not{C}} = \theta_M + \theta_U + \sum_{g=1}^G \theta_g[f^{(m)}_g (1 - p^{(m)}_g) + f^{(u)}_g (1-p^{(u)}_g)],
\]
\end{enumerate}
and so on. 

\section{Estimation}

Of course, we are not so much interested in computing probabilities like those listed above as we are 
with {\em estimating} $\btheta$.  I am going to pursue estimation of $\btheta$ from within the Bayesian
framework because I want to take samples from the posterior, and also because it seems that it will
provide a nice, coherent way of dealing with the many different data sources and the variety of 
latent variables in this estimation problem.

It is important to understand that I will be agglomerating different fishery strata in order to obtain
relatively large samples of CWTs with which to estimate $\btheta$.  In such cases I will assume that
$\btheta$ is identical across all agglomerated strata.  This is obviously a simplification, but it also
simply does not make sense to estimate a separate $\btheta$ for every stratum.  The other consequence
of this policy is that I will likely end up using both visually and electronically detected samples, together, to estimate any particular $\btheta$.  Thus, we would like our estimation method to
appropriately weight information from each type of sample.  I believe the Bayesian framework I
describe below achieves that.

\subsection{Missing data formulation}
We can develop our estimation framework by noticing that what is really missing here is information
that would make the estimation problem quite easy---namely, if we knew the release group of every
fish we sampled (whether it had a CWT or an ad-clip or not).  In that case, we would have a simple multinomial
formulation with a Dirichlet prior.  Of course, we don't know the release group of every sampled
fish---that is missing!  But we can easily formulate a way to sample over those origins given
everything else we know about that fish.  

So, for every fish that is handled in some way (either visually inspected for an ad-clip or
electronically scanned for a CWT, we can sample over that crucial piece of missing data.  

\subsection{The data (missing and otherwise) for each ``sampled'' fish}

For each fish that is sampled (basically someone looks at it to see if it has a mark, or they
run it through a metal detector to see if it beeps, here is what we shall keep track of (and
update/impute as necessary).

\begin{description}
	\item [How sampled?]  It gets a ``V'' if it was visually sampled or an E if electronic.  There should
	not be any NA's here.
	\item [Did it beep?] True/False, only relevant if ``How sampled" is E.  This could be NA but it probably
	shouldn't be.
	\item [Ad-clipped?]  True/False.  An ad-clip, even in combination with any other marks is still counted
	as just an ad-clip.  There can be NAs here. Note that an NA means that the fish was either not
	inspected for a mark, or it was and its marks status was ambiguous.
	\item [CWT status:] This stores the tag\_code of the CWT or other information.  The possibilities are:
	\begin{enumerate}
		\item A tag code
		\item Not present (i.e. not tag found in the fish)
		\item Tag present but not readable (this will be any category like tag-not readable, or tag
		recovered and then lost, etc.
		\item NA --- This signifies that it is unknown if the fish actually had a CWT or not.
	\end{enumerate}

\end{description}


I think that pretty much does it for the missing data.  Given an estimate of $\btheta$ and the $f$'s and $p$'s we should be able to compute the joint posterior for all of these possibilities, which means
we can totally sample over them.  Cool!






There are several different sampling scenarios that must be accommodated.
\subsection{Visual sampling only}
In this scenario, $N_t$ fish are inspected in the fishery and $N_m$ of those are found to have
ad-clips (\ie are found to be marked).  Then, a number $s_m$ of the $N_m$ marked fish are
processed for CWTs.  Let $n_g$ be the number of CWTs recovered from release group $g$ in this 
manner, so that, of course, $s_m = \sum_{g=1}^G n_g$. Denote the whole collection of $n_g$'s by $\bm{n}$.  In this case, it appears to me that the
likelihood for $\theta$ can be written down as:
\[
p(N_m, \bm{n} | \btheta) = \psi_M ^ {N_m} \psi_U ^ {N_t - N_m} \prod_{g=1}^G (\theta_g f^{(m)}_g
p^{(m)}_g)^{n_g}
\]
though that is not complete because it needs to have a term for the marked fish with no coded wire
tag.  Hmm...this looks good though.  I think we can deal with all of these thing through Gibbs
sampling of a missing data model where the the missing data is the release group of origin of
each fish.  Clearly, with an estimate of $\btheta$ and knowing the $f$'s and $p$'s we can compute
a full conditional for the origin of each fish in the sample, whether they had a CWT or not...

I think that will be the way forward.



\section{Other}
While $\btheta$ may be what we want to estimate, typically it is easier to estimate quantities
that we will denote by $\phi$ which are conditional probabilities associated with $\btheta$.
For example, consider what we will call $\phi_g^{(m)}$---the fraction of fish, from amongst all those
which carry a CWT, and are marked, that come from release group $g$.   




\bibliography{../bibstuff/pbtfeas}
\bibliographystyle{../bibstuff/men}
 \end{document}
