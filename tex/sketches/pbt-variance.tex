\documentclass[11pt]{article}
\usepackage{graphicx}
\usepackage{amssymb}
\usepackage{epstopdf}
\usepackage{amsfonts}
\usepackage{natbib}
\usepackage{subfigure}
\usepackage{pdfsync}
\usepackage{xspace}
%\usepackage[pdftex, colorlinks=true,urlcolor=blue]{hyperref}
%\usepackage{wallpaper}

%%%% HERE IS SOME WATERMARK STUFF
%\addtolength{\wpXoffset}{-.2in}
%\CenterWallPaper{1.1}{/Users/eriq/Documents/work/nonprj/WaterMarks/StrictDraftWatermark.eps}


\DeclareGraphicsRule{.tif}{png}{.png}{`convert #1 `dirname #1`/`basename #1 .tif`.png}

%% My baseurl.  Can be changed as necessary.


%% some handy things for making bold math
\def\bm#1{\mathpalette\bmstyle{#1}}
\def\bmstyle#1#2{\mbox{\boldmath$#1#2$}}
\newcommand{\thh}{^\mathrm{th}}


%% Some pretty etc.'s, etc...
\newcommand{\cf}{{\em cf.}\xspace }
\newcommand{\eg}{{\em e.g.},\xspace }
\newcommand{\ie}{{\em i.e.},\xspace }
\newcommand{\etal}{{\em et al.}\ }
\newcommand{\etc}{{\em etc.}\@\xspace}


\newcommand{\CR}{\mathrm{CR}}
\newcommand{\CWT}{\mathrm{CWT}}
\newcommand{\PBT}{\mathrm{PBT}}
\newcommand{\Var}{\mathrm{Var}}


%% the page dimensions from TeXShop's default---very nice
\textwidth = 6.5 in
\textheight = 9 in
\oddsidemargin = 0.0 in
\evensidemargin = 0.0 in
\topmargin = 0.0 in
\headheight = 0.0 in
\headsep = 0.0 in
\parskip = 0.2in
\parindent = 0.0in


\title{Variance of PBT-based estimators}
\author{Eric C. Anderson\thanks{
    Fisheries Ecology Division, 
    Southwest Fisheries Science Center, 
    110 Shaffer Road,
    Santa Cruz, CA 95060}
}
\begin{document}

\maketitle

\begin{abstract}
Here I've written down some expressions developed for the variance of estimators of contribution rate from PBT that
take account of the variance in family size.  This is taken from several things I prepared a while ago while collaborating
on a grant proposal with Shawn Narum and others.
\end{abstract}


Coded wire tag (CWT) data are the primary source of information used to manage Chinook salmon stocks and 
fisheries coast wide. Tag recoveries in fisheries, in hatcheries, and on spawning grounds provide the data needed 
to estimate stock-specific mortalities and distributions in ocean fisheries, and the impacts of fishery management 
actions. Of particular importance is the ability to estimate fishery impacts on stocks of low abundance, as impacts 
to these stocks typically drive fishery management decisions. Reductions in ocean fisheries combined with low 
abundance can make detecting fish from these stocks a rare event. Increasing tag numbers in hatchery indicator 
stocks or sampling fisheries at higher rates can offset these factors to some degree but have their limits.
One of the many advantages of the PBT approach is the greatly reduced cost of tagging relative to CWTs. 
Accordingly, PBT may be well-suited to increasing the tagged fraction of low-abundance stocks in a cost-effective
manner. Higher tagging rates with PBT would enable more precise estimates of the contribution of low-abundance
stocks in ocean fisheries than current CWT programs (at least for stocks that are not already tagged 
at a 100\% rate). To quantify the potential gains from using PBT, however, we will need to develop an understanding of the
 precision of contribution rate and exploitation rate estimators using CWTs or PBT.
 
 
The contribution rate, $\CR_{ai}$, is the proportion of fish from stock $i$ present in the ocean from whence the landed 
fish in fishery $a$ are taken. If it were possible to count the number of fish from stock $i$ harvested in fishery $a$, $\CR_{ai}$ 
could be estimated as
\begin{equation}
\widehat{\CR}_{ai} = \frac{C_{ai}}{C_a}
\label{eq:crai-1}
\end{equation}
where,
\begin{eqnarray*}
\widehat{\CR}_{ai} & = & \mbox{the estimated contribution rate of fish from stock}~i~\mbox{in fishery}~a \\
C_{ai}                    & = &  \mbox{the number of fish from stock}~i~\mbox{landed in fishery}~a \\
C_a                       & = &  \mbox{the total catch in fishery}~a.
\end{eqnarray*}


Of course, $C_{ai}$ is typically not known, as it is usually difficult to distinguish fish from different stocks. 
Therefore it is customary to modify (\ref{eq:crai-1}) by using an estimate of $C_{ai}$, \ie 
\begin{equation}
\widehat{\CR}_{ai} = \frac{\widehat{C}_{ai}}{C_a}
\label{eq:crai-2}
\end{equation}
with  $\widehat{C}_{ai}$ being estimated using fish known to be from stock $i$ through some sort of tagging
effort.  In the case of CWTs  $C_{ai}$ can be estimated by:
\begin{equation}
\widehat{C}_{ai} = \frac{\CWT_{ai}}{s_a\theta_i}.
\end{equation}
where $\CWT_{ai}$ is the number of CWT recoveries from stock $i$ in fishery $a$, $s_a$ is the
fraction of the fishery sampled 
for CWTs, and $\theta_i$ is the tag fraction of stock $i$ at release. Often, tag fractions are
cohort specific, but for now we will assume a constant tag fraction across all brood 
years represented.


Using the model implicit in \citet{Ber&Cla1996}'s large-sample variance approximation for
their estimator of $C_{ai}$ (called $\hat{r}_{ai}$ in their equation 6), the variance of
$\widehat{\CR}_{ai}$ can be derived as:
\begin{equation}
\Var(\widehat{\CR}_{ai}) = 
\frac{ \CR_{ai} (1 - \CR_{ai})}{C_a} +
\frac{C_{ai}(1 - s_a\theta_i)}{C_a s_a \theta_i}
\label{eq:cwt-var}
\end{equation}

Once  a tag is decoded, hatchery origin and age are assumed to be known without error. This equation can be 
used to predict the precision of CWT-based estimates of contribution rate for any given true values of $\CR_{ai}$, and 
known values of $s_a$ and $\theta_i$. It shows that, for a given contribution rate, larger tagging and sampling fractions will 
result in smaller variances.


A similar approach can be taken to derive the precision of estimates of the contribution rate when the tagging of 
fish is done via PBT, but the tagging rate, $\theta_i$, must be handled a little differently. Using PBT, the tagging rate is a 
function of two things: 1) $A_i$, the fraction of parent pairs from stock $i$ included in the parent data base, and 2) $\mu_i$, 
the probability that an individual with both parents in the data base is not successfully assigned to its parents 
(i.e., the ?false negative rate?). A straightforward calculation gives $\theta_i = A_i(1-\mu_i)$. Currently, $\mu_i$
is not well studied for 
most hatcheries, though work in our lab suggests that it is on the order of 3\% for the hatcheries we have 
studied closely. Methods are in 
development that allow uncertainty in parental assignments to be integrated into the estimation, which will, anyway, 
reduce the effect of uncertainty in $\mu_i$ on the variance of $\CR_{ai}$. Nonetheless, a reasonable range
to expect for $\mu_i$ is about 1\% to 10\%.

Unfortunately, $A_i$ cannot be observed without some error. By keeping careful track of the number of fish spawned 
each day at a hatchery, and the number of genotypes successfully obtained from those fish, it is possible to 
calculate $E_i$---the expected value of $A_i$---but the actual value is a random variable that reflects the fact that there 
is variance in the size of different families in a salmon population.
Accordingly, with PBT, an unbiased estimator for the contribution rate is
\begin{equation}
\widehat{\CR}_{ai} = \frac{\PBT_i }{C_a s_a E_i(1-\mu_i)}
\end{equation}
where $\PBT_i$ is the number of fish identified to stock $i$ using PBT. When the variance of this estimator is derived 
using the same assumptions that lead to (\ref{eq:cwt-var}) for CWTs
we find, a similar-looking expression, with a few extra terms for the dependence on the
variance of $A_i$:
\begin{equation}
\Var(\widehat{\CR_{ai}}) =  \frac{\CR_{ai}[1-\CR_{ai}(\sigma^2_A + E_i^2)E_i^{-2}]}{C_a} + \frac{\CR_{ai}}{C_a s_a E_i (1-u_i)}\biggl(1-s_a E_i (1-u_i)\biggr) + \CR_{ai}^2 \frac{\sigma^2_A}{E^2_i},
\end{equation}
where $\sigma^2_A$ is the variance of $A_i$.  Note that setting $E_i(1-u_i)=\theta_i$
and $\sigma^2_A=0$ recovers the original expression for CWTs.


\bibliography{../bibstuff/pbtfeas}
\bibliographystyle{../bibstuff/men}
 \end{document}
